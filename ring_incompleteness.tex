% ------------------------------------------------------------------------
% bjourdoc.tex for birkjour.cls*******************************************
% ------------------------------------------------------------------------
%%%%%%%%%%%%%%%%%%%%%%%%%%%%%%%%%%%%%%%%%%%%%%%%%%%%%%%%%%%%%%%%%%%%%%%%%%

\documentclass{birkjour}
%
%
% THEOREM Environments (Examples)-----------------------------------------
%
 \newtheorem{thm}{Theorem}[section]
 \newtheorem{cor}[thm]{Corollary}
 \newtheorem{lem}[thm]{Lemma}
 \newtheorem{prop}[thm]{Proposition}
 \theoremstyle{definition}
 \newtheorem{defn}[thm]{Definition}
 \theoremstyle{remark}
 \newtheorem{rem}[thm]{Remark}
 \newtheorem*{ex}{Example}
 \numberwithin{equation}{section}


% \newcommand{\abs}{\lvert{#1}\rvert}

\newcommand{\mcL}{\mathcal{L}}
\newcommand{\mcH}{\mathcal{H}}
\newcommand{\bbT}{\mathbb{T}}
\newcommand{\bbC}{\mathbb{C}}
\newcommand{\bbD}{\mathbb{D}}

\newcommand{\eexp}[1]{e^{#1}}
\newcommand{\iu}{{i\mkern1mu}}
\renewcommand{\Re}{\operatorname{Re}}
\renewcommand{\Im}{\operatorname{Im}}
\renewcommand{\phi}{\varphi}

\begin{document}

%-------------------------------------------------------------------------
% editorial commands: to be inserted by the editorial office
%
%\firstpage{1} \volume{228} \Copyrightyear{2004} \DOI{003-0001}
%
%
%\seriesextra{Just an add-on}
%\seriesextraline{This is the Concrete Title of this Book\br H.E. R and S.T.C. W, Eds.}
%
% for journals:
%
%\firstpage{1}
%\issuenumber{1}
%\Volumeandyear{1 (2004)}
%\Copyrightyear{2004}
%\DOI{003-xxxx-y}
%\Signet
%\commby{inhouse}
%\submitted{March 14, 2003}
%\received{March 16, 2000}
%\revised{June 1, 2000}
%\accepted{July 22, 2000}
%
%
%
%---------------------------------------------------------------------------
%Insert here the title, affiliations and abstract:
%


\title[Incompleteness of resonance states]
 {Incompleteness of resonance states for \\
 quantum ring with two semi-infinite edges}



%----------Author 1
\author[D.A. Gerasimov]{Dmitrii Gerasimov}
\address{
ITMO University\\
Saint Petersburg, Russia
}
\email{the.dmitrii.g@gmail.com}

%----------Author 2
\author[I. Y. Popov]{Igor Popov}
\address{
ITMO University\\
Saint Petersburg, Russia
}
\email{iypopov@corp.ifmo.ru}
%----------classification, keywords, date
\subjclass{Primary 81U20; Secondary 46N50}

\keywords{Scattering; resonance; quantum graph; incompleteness; functional model}

\date{March 12, 2017}
%----------additions
%%% ----------------------------------------------------------------------

\begin{abstract}
In this paper we investigate scattering problem for a quantum graph (a ring), connected
with two semi-infinite leads via a Dirac delta functio at boundary. We prove
incompleteness of the system of resonance states in $L_2$ on finite subgraph and
discuss a relation with the factorization of the characteristic function in Sz-Nagy functional model.
\end{abstract}

%%% ----------------------------------------------------------------------
\maketitle
%%% ----------------------------------------------------------------------
%\tableofcontents
\section{Introduction}

TODO some theory

TODO references to previous papers (the one where completeness was proven)

It is tempting to assume that any finite quantum graph, composed of edges of finite length will have similar properties.
However, we will show that it is not necessarily the case. In this paper we investigate a quantum graph $\Gamma$ on Figure \ref{fig:graph}.

\begin{figure}
\includegraphics[width=\textwidth,keepaspectratio]{figure.eps}
\caption{Quantum graph $\Gamma$ composed of semi-infinite edges $\Omega_L, \Omega_R$ and a resonator $\Omega$, consisting of a ring with circumference $1$.}
\label{fig:graph}
\end{figure}

\subsection{Notations}
We will use the following notations:
\begin{itemize}
\item $\mathbb{C}$: complex plane, $\mathbb{C} = \{ x + \mathrm{i}
y \mid x, y \in \mathbb{R} \}$
\item $\mathbb{H}$: upper complex half-plane, $\mathbb{H} =
\{ x + \mathrm{i} y \mid y > 0, x, y \in \mathbb{R} \}$
\item $\mathbb{D}$: unit disk, $\mathbb{D} = \{ z \mid
\left|z\right| < 1 \}$
\item $\mathbb{T}$: unit circle, $\mathbb{T} = \partial \mathbb{D}
=  \{z \mid \left|z\right| = 1 \}$
\item $z$ denotes a complex argument on complex plane $\mathbb{C}$
\item $\zeta$ denotes a complex argument on unit disk $\mathbb{D}$
\end{itemize}

\subsection{Cayley transform}

Cayley transform maps $\mathbb{H}$ to $\mathbb{D}$:
\[
W(z) = \frac{z - \mathrm{i}}{z + \mathrm{i}},
\]
whereas inverse Cayley transform maps $\mathbb{D}$ to
$\mathbb{H}$:
\begin{equation}\label{eq:cayley_inverse}
w(\zeta) = \mathrm{i} \frac{1 + \zeta}{1 - \zeta}.
\end{equation}

One notable property of the Cayley transform  is that it
injectively maps $\mathbb{R}$ into unit circle $\mathbb{T}$.

Another important property  we are going to use is that Cayley
transform preserves circles. In particular, a circle of radius $r,
0 < r < 1$, centered at zero, under inverse Cayley transform maps
into a circle with center at point $\mathrm{i} C(r)$ and having
the radius $R(r)$, where:

\begin{equation}\label{eq:c_and_r}
\begin{aligned}
   C(r) &= \operatorname{Im} \frac{w(r) + w(-r)}{2} =
   \frac{1 + r^2}{1 - r^2},
\\ R(r) &= \operatorname{Im} \frac{w(r) - w(-r)}{2} =
\frac{2 r}{1 - r^2}.
\end{aligned}
\end{equation}

From these formulas,  it is easy to see that if $r$ approaches
$1$, then $R(r)$ goes to infinity and $C(r)$ converges to $R(r)$.

Also, we'll note two useful facts:
\begin{subequations}
\begin{equation}
C(r) - R(r) = \frac{(1 - r)^2}{1 - r^2} > 0,
\label{eq:cr_positive}
\end{equation}
\begin{equation}
C(r) - R(r) = \frac{1 - r}{1 + r} < \frac{1}{R(r)} = \frac{1 -
r^2}{2 r} \implies C - R \text{\ is\ }
\mathcal{O}\left(\frac{1}{R}\right).
\label{eq:cr_small}
\end{equation}
\end{subequations}


\subsection{S-matrix}\label{sec:smatrix}
Our quantum graph $\Gamma$ has two semi-infinite leads
$\Omega_L,\Omega_R$. To construct the scattering matrix, we
consider two scattering problems for plane waves coming from
$\Omega_L$ and $\Omega_R$, correspondingly:
\begin{equation}\label{eq:wlwl}
\begin{aligned}
   \psi_L(x) &= A e^{\mathrm{i} k x} + B e^{-\mathrm{i} k x}
\\ \psi_R(x) &= C e^{\mathrm{i} k x} + D e^{-\mathrm{i} k x}
\end{aligned}
\end{equation}
The scattering matrix relates the final and the initial states of
the system:
\begin{equation}\label{eq:smatrix}
\begin{pmatrix} B \\ C \end{pmatrix} = S \begin{pmatrix} A \\
D \end{pmatrix},
\end{equation}
and describes scattering properties of the obstacle. In our case
the obstacle is the subgraph $\Omega$. Given $A,D$, one obtains
$B,C$ by substitution of (\ref{eq:wlwl}) into the coupling
conditions at the graph vertices. It
gives us the S-matrix $S(k)$.

\subsection{Completeness criterion}
There is simple criterion for absence of the singular inner factor
for the case  $\dim  N < \infty$ (for general operator case there
is no simple criterion):

\begin{thm}[\cite{Nik}, Lecture 4 ] Let $\dim  N < \infty$. The following
statements are equivalent:

1.  $S$ is a Blaschke-Potapov  product;

2. \begin{equation}\label{eq:crit} \lim\limits_{r \to 1}
\int\limits_{C_r} \ln \left|\det S(k)\right| \frac{2
\mathrm{i}}{(k + i)^2} dk = 0,
\end{equation}
where $C_r$ is an image of $\left|\zeta\right| = r$ under the
inverse Cayley transform (\ref{eq:cayley_inverse}).
\end{thm}

The integration countour can be parameterized as $C_r = \{R(r)
e^{\mathrm{i} t} + \mathrm{i} C(r) \mid t \in [0, 2 \pi)\}$ (see
\ref{eq:c_and_r}). For brevity, define:
\[
s(k) = \left|\det S(k)\right|,
\]
and after throwing away constants which are irrelevant for
convergence, we get the final form of the criterion, which is
convenient for us and will be used afterwards:
\begin{equation}\label{eq:critp}
\lim\limits_{r \to 1} \int\limits_{0}^{2 \pi} \ln s(R(r)
e^{\mathrm{i} t} + \mathrm{i} C(r)) \frac{R}{(R(r) e^{\mathrm{i}
t} + \mathrm{i} C(r) + i)^2} dt = 0.
\end{equation}

\section{Proof of the main theorem}

After solving (\ref{eq:bundle_system}) in conjunction with
(\ref{eq:wlwl}), we get the S-matrix. The expression for $S(k)$ is
rather large. We will not present it here because in the following
consideration we need only its determinant which has essentially
more simple form:
\begin{equation}\label{det-s}
\det S(k) = \frac{2 i \, W \cos\left(k\right) -
{\left(W^{2} + 1\right)} \sin\left(k\right)}{2 i \,
W \cos\left(k\right) + {\left(W^{2} + 1\right)} \sin\left(k\right)}
\end{equation}

Since the system is symmetric around the origin, it is enough to consider scattering from the left or the right only. If the wave with the wavevector $k$ scattered from left to right, we get the wavefunctions in the following form:

\begin{align*}
\psi_L(x) &= \eexp{\iu k x} + R \eexp{-\iu k x} \\
\psi_R(x) &= T \eexp{\iu k x}\\
\psi_\Omega(x) &= P \sin(k x) + Q \cos(k x)
\end{align*}
, where $R$ and $T$ are reflection and transmission coefficient. Since the system is symmetric, scattering matrix takes the form
$S(k) = \begin{pmatrix} R(k) & T(k) \\ T(k) & R(k) \end{pmatrix}$.

At the vertex $V$ we put a Dirac $\delta$-barrier of strength $a$, which implies a continuity condition for the wavefunction and a derivative jump:

\begin{align*}
\psi_L(0) = \psi_R(0) = \psi_\Omega(0) = \psi_\Omega(1) \\ 
-\psi'_L(0) + \psi'_\Omega(0) - \psi'_\Omega(1) + \psi'_R(0) = a \psi_L(0)
\end{align*}


% \section{Вычисление S-матрицы}
To compute transmission and reflection coefficients, we have to solve the following system of linear equations:
\begin{align*}
& 1 + R &= T \\
& 1 + R &= Q \\
& Q \cos k + P \sin k &= T \\
& -P k \cos k + Q k \sin k + P k + \iu R k + \iu T k - \iu k &= T a
\end{align*}
, which yields:
\begin{align*}
R(k) = -\frac{2 \, k \cos\left(k\right) + a \sin\left(k\right) - 2 \, k}{2 \, k \cos\left(k\right) + {\left(a - 2 i \, k\right)} \sin\left(k\right) - 2 \, k} \\
T(k) = -\frac{2 i \, k \sin\left(k\right)}{2 \, k \cos\left(k\right) + {\left(a - 2 i \, k\right)} \sin\left(k\right) - 2 \, k}
\end{align*}
, and after substituting these in the S-matrix, we get:
\[
\det S = 
\frac
{\cos\left(k\right) + {\left(\frac{a}{2 k} + i\right)} \sin\left(k\right) - 1}
{\cos\left(k\right) + {\left(\frac{a}{2 k} - i\right)} \sin\left(k\right) - 1}
\]


% \section{Исследование полноты при $a=0$}
Let's consider $a=0$:
\[
\det S
= \frac
{\cos\left(k\right) + \iu \sin\left(k\right) - 1}
{\cos\left(k\right) - \iu \sin\left(k\right) - 1}
= \frac{\eexp{\iu k} - 1}{\eexp{-\iu k} - 1}
= -\eexp{i k}
\]

\[
\ln \left|{\det S}\right| = \ln \eexp{- \Im k} = -\Im k
\]

We will compute this integral in the space of the unit disk, in order to do that, we apply the inverse Cayley transform $k \to \iu \frac{1 + \zeta}{1 - \zeta}$ to the integrand: $\Im k \to \Im \left( \iu \frac{1 + \zeta}{1 - \zeta} \right) $.

\[
  \lim\limits_{r = 1} \int\limits_{\left|\zeta\right| = r} \ln \left|\det S(\zeta)\right| d \zeta
= \lim\limits_{r = 1} \int\limits_{\left|\zeta\right| = r} \Im \left( \iu \frac{1 + \zeta}{1 - \zeta} \right)  d\zeta = \dots
\]
, make a coordinate change into polar: $\zeta \to r \eexp{\iu \phi}, d\zeta \to r \iu \eexp{\iu \phi}$:
\[
\dots = \lim\limits_{r = 1} \int\limits_{\left|\zeta\right| = r} \Im \left( \iu \frac{1 + r \eexp{\iu \phi}}{1 - r \eexp{\iu \phi}} \right) r \iu \eexp{\iu \phi} d\phi
\]

Compex integral is by the definition a sum of the integrals of the real part and the imaginary part of the integrand. Let's evaluate the imaginary one:
\begin{align*}
\Im \left(  \Im \left( \iu \frac{1 + r \eexp{\iu \phi}}{1 - r \eexp{\iu \phi}} \right) r \iu \eexp{\iu \phi} \right)
 &= r \Re \left(  \Re \left( \frac{1 + r \eexp{\iu \phi}}{1 - r \eexp{\iu \phi}} \right) \eexp{\iu \phi} \right) \\
 &= r \Re \left( \frac{1 + r \eexp{\iu \phi}}{1 - r \eexp{\iu \phi}} \right) \Re \left(   \eexp{\iu \phi} \right) \\
 &= r \Re \left( \frac{(1 + r \eexp{\iu \phi}) (1 - r \eexp{-\iu \phi}) }{(1 - r \eexp{\iu \phi}) (1 - r \eexp{-\iu \phi})} \right) \cos \phi \\
 &= r \Re \left( \frac{1 - r^2 + 2 \iu r \sin \phi}{1 + r^2 - 2 r \cos \phi} \right) \cos \phi \\
 &= r \frac{1 - r^2}{1 + r^2 - 2 r \cos \phi} \cos \phi \\
 % TODOD ???? WTF IS GOING ON HERE???
 &= 2 \pi r^2
\end{align*}

One can see that as $r \to 1$, the integral of the imaginary part goes to $2 \pi$, therefore, according to the completeness criterion, the system of resonant states of $\Gamma$ is not complete on the ring $\Omega$.

% ------------------------------------------------------------------------

\subsection*{Acknowledgment}
TODO ?????


\begin{thebibliography}{1}
\bibitem{Nik}
N.Nikol'skii. {\textit Treatise on the shift operator: spectral
function theory}. Springer Science \& Business Media; Berlin,
(2012).

\bibitem{test} A. B. C. Test, \textit{On a Test.} J. of Testing
\textbf{88} (2000), 100--120.
\bibitem{latex} G. Gr\"atzer, \textit{Math into \LaTeX.} 3rd Edition,
Birkh\"auser, 2000.
\end{thebibliography}

% ------------------------------------------------------------------------
\end{document}
% ------------------------------------------------------------------------
